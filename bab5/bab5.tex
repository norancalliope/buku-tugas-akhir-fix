\chapter{PENUTUP}
\label{sec:chap5_tutup}
\vspace{1ex}
\section*{}
Pada bab ini akan dipaparkan kesimpulan dari hasil pengujian yang telah dilakukan. Kesimpulan ini akan menjawab permasalahan yang diangkat dengan merujuk pada tujuan dari pelaksanaan tugas akhir ini. Pada bab ini juga dipaparkan saran mengenai hal yang dapat dilakukan untuk mengembangkan penelitian dengan topik yang sama. \vspace{1ex}

\section{Kesimpulan}
\label{sec:sec4_kesimpulan}
\vspace{1ex}
Berdasarkan hasil pengujian yang telah dilakukan dalam tugas akhir ini, dapat diambil kesimpulan bahwa sistem kendali kursi roda berbasis SIBI dan \emph{braking system} menggunakan LSTM dan YOLOv11 dapat berjalan dengan baik. Sistem dapat diimplementasikan pada laptop dan NUC dengan performa yang baik. Pengujian model YOLOv11 dengan epoch 100 lebih bagus sekitar 0.5\%–1\%. Model LSTM dapat memiliki akurasi 100\% dalam pemanggilan kelas. Pengujian FPS dengan hasil terbaik adalah pengujian dengan YOLOv11 pada laptop dengan rata-rata FPS yang didapat sebesar 24,969. Pengujian \emph{respond time} LSTM rata-rata berada di xxxxxxxx ms. Pengujian kesesuaian jarak dengan eror paling kecil adalah pengujian dengan model YOLOv11 dengan jarak 150 cm dengan eror 0.73\%. Performa pengereman otomatis terbaik adalah model YOLOv11 dengan akurasi 100\%.
\section{Saran}
\label{sec:sec4_saran}
\vspace{1ex}
Berdasarkan hasil yang diperoleh dari tugas akhir ini, saran yang dapat dipertimbangkan untuk pengembang lebih lanjut adalah sebagai berikut:
\begin{enumerate}
    \item Menambahkan kelas yang lebih beragam pada YOLO agar sistem pengereman lebih \emph{safety}.
    \item Menambahkan gestur yang lebih bervariasi selain SIBI untuk memperluas opsi sistem kendali kursi roda.
    \item Mempertimbangkan penggunaan metode lain CNN-LSTM untuk ekstraksi fitur dalam bentuk citra sehingga dapat melakukan klasifikasi gerakan bahasa isyarat dnegan lebih baik dan akurat lagi.
\end{enumerate}

