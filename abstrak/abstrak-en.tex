\chapter*{ABSTRACT}
\begin{center}
  \large
  \textbf {DEVELOPMENT OF WHEELCHAIR CONTROL SYSTEM BASED ON SIBI AND BRAKING SYSTEM USING LSTM AND YOLOv11}
\end{center}
% Menyembunyikan nomor halaman
\thispagestyle{empty}

\begin{flushleft}
  \setlength{\tabcolsep}{0pt}
  \bfseries
  \begin{tabular}{lc@{\hspace{6pt}}l}
  Student Name / NRP&: &I Putu Deva Febriana /  5024211016\\
  Department&: &Computer Engineering FTEIC - ITS\\
  Advisor&: &1. Dr. Supeno Mardi Susiki Nugroho,S.T.,M.T.\\
  & & 2. Dr. Eko Mulyanto Yuniarno,S.T.,M.T.\\
  \end{tabular}
  \vspace{4ex}
\end{flushleft}
\textbf{Abstract}

% Isi Abstrak
This research aims to develop a wheelchair control system based on Indonesian Sign Language (SIBI) gestures and a auto braking system using YOLOv11 and Long Short-Term Memory (LSTM). The system is designed to provide an innovative solution for wheelchair users, particularly those with physical and communication limitations, by utilizing SIBI hand gestures as commands to move and control the wheelchair. YOLOv11 technology is employed to detect object around the user of wheelchair, while LSTM is applied to process gesture sequences and predict the user's intended commands using SIBI gestures. Additionally, the automatic braking system is developed to enhance user safety by utilizing a camera to detect obstacles or emergency situations. the testing shows that the proposed system has high accuracy and responsiveness, which can improve the independence and mobility of wheelchair users.

\vspace{2ex}
\noindent
\textbf{Keywords: \emph{Intelligent wheelchair, SIBI gesture, YOLOv11, LSTM, Braking System.}}